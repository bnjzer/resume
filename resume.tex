%% start of file `template.tex'.
%% Copyright 2006-2015 Xavier Danaux (xdanaux@gmail.com).
%
% This work may be distributed and/or modified under the
% conditions of the LaTeX Project Public License version 1.3c,
% available at http://www.latex-project.org/lppl/.


\documentclass[11pt,a4paper,sans]{moderncv}        % possible options include font size ('10pt', '11pt' and '12pt'), paper size ('a4paper', 'letterpaper', 'a5paper', 'legalpaper', 'executivepaper' and 'landscape') and font family ('sans' and 'roman')

\moderncvstyle{banking}                             % style options are 'casual' (default), 'classic', 'banking', 'oldstyle' and 'fancy'
\moderncvcolor{black}                               % color options 'black', 'blue' (default), 'burgundy', 'green', 'grey', 'orange', 'purple' and 'red'
%\renewcommand{\familydefault}{\sfdefault}         % to set the default font; use '\sfdefault' for the default sans serif font, '\rmdefault' for the default roman one, or any tex font name
\nopagenumbers{}                                  % uncomment to suppress automatic page numbering for CVs longer than one page

% adjust the page margins
%\usepackage[scale=0.75]{geometry}
\usepackage[vmargin=1.0cm,hmargin=1.5cm]{geometry}
%\setlength{\hintscolumnwidth}{3cm}                % if you want to change the width of the column with the dates
%\setlength{\makecvheadnamewidth}{10cm}            % for the 'classic' style, if you want to force the width allocated to your name and avoid line breaks. be careful though, the length is normally calculated to avoid any overlap with your personal info; use this at your own typographical risks...

% personal data
\name{Benjamin}{BENOIST}
\title{Data Engineer}                               % optional, remove / comment the line if not wanted
%\address{}{}{Paris, FRANCE}% optional, remove / comment the line if not wanted; the "postcode city" and "country" arguments can be omitted or provided empty
\phone[mobile]{+33~7~77~36~73~47}                   % optional, remove / comment the line if not wanted; the optional "type" of the phone can be "mobile" (default), "fixed" or "fax"
%\phone[fixed]{+2~(345)~678~901}
\email{benjamin.benoist@insa-rouen.fr}                               % optional, remove / comment the line if not wanted
%\homepage{www.johndoe.com}                         % optional, remove / comment the line if not wanted
\social[linkedin]{benjaminbenoist}                        % optional, remove / comment the line if not wanted
%\social[twitter]{jdoe}                             % optional, remove / comment the line if not wanted
\social[github]{bnjzer}                              % optional, remove / comment the line if not wanted
%\extrainfo{additional information}                 % optional, remove / comment the line if not wanted
%\photo[64pt][0.4pt]{picture}                       % optional, remove / comment the line if not wanted; '64pt' is the height the picture must be resized to, 0.4pt is the thickness of the frame around it (put it to 0pt for no frame) and 'picture' is the name of the picture file
% \quote{"There are only 10 types of people in the world: those who understand binary, and those who don't."}                                 % optional, remove / comment the line if not wanted

%----------------------------------------------------------------------------------
%            content
%----------------------------------------------------------------------------------

\begin{document}

%-----       resume       ---------------------------------------------------------

% phrases les plus courtes possibles, et pas trop techniques
% emploi: bien décrire les rôles à chaque fois
% utliser des verbes d'action
% key achievements
% core skills utiles pour le post
% interests: tout ce qui nous différencie

\makecvtitle

\section{Experience}
\vspace{0.7em}

  \cventry{September 2012 - Now}{Data Engineer}{French Ministry of Defense}{PARIS}{}{
\begin{itemize}%
  \item Built a stream processing platform with \textit{Zookeeper}, \textit{Kafka}, \textit{Storm} and \textit{Java} (lead developer in a team of 4 engineers)
  \begin{itemize}%
    \item Used Gitlab to develop, with issues and merge requests;
    \item Packaged for \textit{Debian} with bash scripts;
    \item Set up a testing environment with \textit{Docker};
    \item Deployed the clusters with \textit{Puppet};
    \item Used \textit{Nagios} to supervise the clusters;
    \item Provided metrics using ELK (\textit{Logstash}, \textit{Elasticsearch} and \textit{Kibana});
    \item Developed a tool to manage the clusters in \textit{Python};
    \item Gained time management skills from creating and then adjusting the planning.
  \end{itemize}
\item Now building another stream processing platform with \textit{Zookeeper}, \textit{Kafka}, \textit{Spark} and \textit{Scala} (\textit{Akka})
\end{itemize}}
  
\vspace{0.2em}

\cventry{February - August 2012}{6 months engineering internship}{French Ministry of Defense}{PARIS}{}{Developped modules in \textit{C} to automate network protocols analysis}

\vspace{0.2em}

\cventry{2010 -- 2011}{Projects supervisor}{INSA's Junior Enterprise}{Rouen}{}{Supervised the project managers, valided the contracts and managed contract amendments}

\vspace{0.2em}

\cventry{2009 (6 months)}{Project manager}{INSA's Junior Enterprise}{Rouen}{}{Established planning, costs and managed a team of 3 people to create a website for a company}

\section{Education}
\vspace{0.7em}

\cventry{August 2011 -- January 2012}{ERASMUS semester}{Royal Institute of Technology (KTH)}{Stockholm}{}{Full TCP/IP stack and routing protocols (RIP, OSPF, BGP). Network programming}  % arguments 3 to 6 can be left empty

\vspace{0.2em}

\cventry{2007 -- 2012}{Master’s Degree in Computer Science}{National Institute of Applied Sciences (INSA)}{Rouen (France)}{}{Linux operating systems, networking (DHCP, DNS, NAT, Firewall), advanced programming and project management}

\section{Languages}
\vspace{0.7em}

\cvitemwithcomment{French}{Native speaker}{}
\cvitemwithcomment{English}{Full professional proficiency}{890/990 at TOEIC before spending one semester abroad}
\cvitemwithcomment{German}{Limited working proficiency}{}

\section{Interests}
\vspace{0.7em}

\cvitem{Sport}{Training for an IronMan}
\cvitem{Travelling}{At least 3 trips abroad per year for the last 4 years}
\cvitem{Nature}{Love hiking and hanging around in wild places}

\clearpage
%-----       letter       ---------------------------------------------------------
% recipient data
%\recipient{Company Recruitment team}{Company, Inc.\\123 somestreet\\some city}
%\date{January 01, 1984}
%\opening{Dear Sir or Madam,}
%\closing{Yours faithfully,}
%\enclosure[Attached]{curriculum vit\ae{}}          % use an optional argument to use a string other than "Enclosure", or redefine \enclname
%\makelettertitle
%
%Lorem ipsum dolor sit amet, consectetur adipiscing elit. Duis ullamcorper neque sit amet lectus facilisis sed luctus nisl iaculis. Vivamus at neque arcu, sed tempor quam. Curabitur pharetra tincidunt tincidunt. Morbi volutpat feugiat mauris, quis tempor neque vehicula volutpat. Duis tristique justo vel massa fermentum accumsan. Mauris ante elit, feugiat vestibulum tempor eget, eleifend ac ipsum. Donec scelerisque lobortis ipsum eu vestibulum. Pellentesque vel massa at felis accumsan rhoncus.
%
%Suspendisse commodo, massa eu congue tincidunt, elit mauris pellentesque orci, cursus tempor odio nisl euismod augue. Aliquam adipiscing nibh ut odio sodales et pulvinar tortor laoreet. Mauris a accumsan ligula. Class aptent taciti sociosqu ad litora torquent per conubia nostra, per inceptos himenaeos. Suspendisse vulputate sem vehicula ipsum varius nec tempus dui dapibus. Phasellus et est urna, ut auctor erat. Sed tincidunt odio id odio aliquam mattis. Donec sapien nulla, feugiat eget adipiscing sit amet, lacinia ut dolor. Phasellus tincidunt, leo a fringilla consectetur, felis diam aliquam urna, vitae aliquet lectus orci nec velit. Vivamus dapibus varius blandit.
%
%Duis sit amet magna ante, at sodales diam. Aenean consectetur porta risus et sagittis. Ut interdum, enim varius pellentesque tincidunt, magna libero sodales tortor, ut fermentum nunc metus a ante. Vivamus odio leo, tincidunt eu luctus ut, sollicitudin sit amet metus. Nunc sed orci lectus. Ut sodales magna sed velit volutpat sit amet pulvinar diam venenatis.
%
%Albert Einstein discovered that $e=mc^2$ in 1905.
%
%\[ e=\lim_{n \to \infty} \left(1+\frac{1}{n}\right)^n \]
%
%\makeletterclosing

\end{document}
