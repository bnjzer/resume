%% start of file `template.tex'.
%% Copyright 2006-2015 Xavier Danaux (xdanaux@gmail.com).
%
% This work may be distributed and/or modified under the
% conditions of the LaTeX Project Public License version 1.3c,
% available at http://www.latex-project.org/lppl/.


\documentclass[11pt,a4paper,sans]{moderncv}        % possible options include font size ('10pt', '11pt' and '12pt'), paper size ('a4paper', 'letterpaper', 'a5paper', 'legalpaper', 'executivepaper' and 'landscape') and font family ('sans' and 'roman')

\moderncvstyle{banking}                             % style options are 'casual' (default), 'classic', 'banking', 'oldstyle' and 'fancy'
\moderncvcolor{black}                               % color options 'black', 'blue' (default), 'burgundy', 'green', 'grey', 'orange', 'purple' and 'red'
%\renewcommand{\familydefault}{\sfdefault}         % to set the default font; use '\sfdefault' for the default sans serif font, '\rmdefault' for the default roman one, or any tex font name
\nopagenumbers{}                                  % uncomment to suppress automatic page numbering for CVs longer than one page

% adjust the page margins
%\usepackage[scale=0.75]{geometry}
\usepackage[vmargin=1.0cm,hmargin=1.5cm]{geometry}
%\setlength{\hintscolumnwidth}{3cm}                % if you want to change the width of the column with the dates
%\setlength{\makecvheadnamewidth}{16cm}            % for the 'classic' style, if you want to force the width allocated to your name and avoid line breaks. be careful though, the length is normally calculated to avoid any overlap with your personal info; use this at your own typographical risks...

%
%\newcommand{\myTitleName}{Software Engineer}
\newcommand{\myTitleName}{Data Engineer}
%\renewcommand*{\namefont}{\fontsize{29}{31}\mdseries\upshape}
%\renewcommand*{\titlefont}{\fontsize{22}{24}\mdseries\upshape}

% personal data
\name{Benjamin}{BENOIST}
\title{\myTitleName}
%\address{}{}{Paris, FRANCE}% optional, remove / comment the line if not wanted; the "postcode city" and "country" arguments can be omitted or provided empty
\phone[mobile]{+33~7~77~36~73~47}                   % optional, remove / comment the line if not wanted; the optional "type" of the phone can be "mobile" (default), "fixed" or "fax"
%\phone[fixed]{+2~(345)~678~901}
\email{benjamin.benoist@insa-rouen.fr}                               % optional, remove / comment the line if not wanted
%\homepage{www.johndoe.com}                         % optional, remove / comment the line if not wanted
\social[linkedin]{benjaminbenoist}                        % optional, remove / comment the line if not wanted
%\social[twitter]{jdoe}                             % optional, remove / comment the line if not wanted
\social[github]{bnjzer}                              % optional, remove / comment the line if not wanted
%\extrainfo{additional information}                 % optional, remove / comment the line if not wanted
%\photo[64pt][0.4pt]{picture}                       % optional, remove / comment the line if not wanted; '64pt' is the height the picture must be resized to, 0.4pt is the thickness of the frame around it (put it to 0pt for no frame) and 'picture' is the name of the picture file
% \quote{"There are only 10 types of people in the world: those who understand binary, and those who don't."}                                 % optional, remove / comment the line if not wanted


%----------------------------------------------------------------------------------
%            content
%----------------------------------------------------------------------------------

\begin{document}

%-----       resume       ---------------------------------------------------------

% phrases les plus courtes possibles, et pas trop techniques
% emploi: bien décrire les rôles à chaque fois
% utliser des verbes d'action
% key achievements
% core skills utiles pour le post
% interests: tout ce qui nous différencie

\makecvtitle

\section{Professional experience}
\vspace{0.7em}

\cventry{September 2012 - Now}{\myTitleName}{French Ministry of Defense}{PARIS}{}{
\begin{itemize}%
  \item Built a stream processing platform in \textit{Java} with \textit{Storm}, \textit{Kafka} and \textit{Zookeeper} (lead developer in a team of 4 engineers)
  \begin{itemize}%
    \item Used Gitlab to develop, with issues and merge requests;
    \item Packaged for \textit{Debian} with \textit{bash} scripts;
    \item Set up a testing environment with \textit{Docker};
    \item Deployed the clusters with \textit{Puppet};
    \item Used \textit{Nagios} to supervise the clusters;
    \item Provided metrics using ELK (\textit{Logstash}, \textit{Elasticsearch} and \textit{Kibana});
    \item Developed a tool to manage the clusters in \textit{Python};
    \item Gained time management skills from creating and then adjusting the planning.
  \end{itemize}
\item Now building another stream processing platform in \textit{Scala} (\textit{Akka Streams}) with \textit{Spark}, \textit{Kafka} and  \textit{Zookeeper}
\end{itemize}}
  
\vspace{0.2em}

\cventry{February - August 2012}{6 months engineering internship}{French Ministry of Defense}{PARIS}{}{Developped modules in \textit{C} to automate network protocols analysis}

\vspace{0.2em}

%\cventry{2010 -- 2011}{Projects supervisor}{INSA's Junior Enterprise}{Rouen}{}{Supervised the project managers, valided the contracts and managed contract amendments}
\cventry{2010 -- 2011}{Projects supervisor}{INSA's Junior Enterprise}{Rouen}{}{Developed processes so that the association can be become a certified Junior Enterprise}

\vspace{0.2em}

%\cventry{2009 (6 months)}{Project manager}{INSA's Junior Enterprise}{Rouen}{}{Established planning, costs and managed a team of 3 people to create a website for a company}

\section{Personal experience}
\vspace{0.7em}

\begin{itemize}
  \item Complete a triangle challenge in \textit{Scala} \href{https://github.com/bnjzer/triangleChallenge}{\faGithub}
  \item Developed a program in \textit{C} that lists duplicated files in a directory (or 2) recursively based on their md5sum \href{https://github.com/bnjzer/findDuplicatesByHash}{\faGithub}
  \item Currently implementing data structures and algorithms in \textit{Scala} \href{https://github.com/bnjzer/dataStructsAlgos}{\faGithub}
  \item Completed Kernighan and Ritchie's book's exercices \href{https://github.com/bnjzer/c}{\faGithub}
  \item Have been going to \href{https://fosdem.org}{\faExternalLink{} FOSDEM} (software developers' meeting) every year for the last 4 years
\end{itemize}

\section{Education}
\vspace{0.7em}

\cventry{August 2011 -- January 2012}{ERASMUS semester}{Royal Institute of Technology (KTH)}{Stockholm}{}{Advanced network programming (mostly \textit{Java}), full TCP/IP stack and routing protocols (RIP, OSPF, BGP)}  % arguments 3 to 6 can be left empty

\vspace{0.2em}

\cventry{2007 -- 2012}{Master’s Degree in Computer Science}{National Institute of Applied Sciences (INSA)}{Rouen (France)}{}{Advanced programming (\textit{C} and \textit{Java}), Linux operating systems, networking (DHCP, DNS, NAT, Firewall)}

\section{Languages}
\vspace{0.7em}

\cvitemwithcomment{French}{Native speaker}{}
\cvitemwithcomment{English}{Full professional proficiency}{}
\cvitemwithcomment{German}{Limited working proficiency}{}

\section{Interests}
\vspace{0.7em}

\cvitem{Sport}{Training for an IronMan}
\cvitem{Travelling}{At least 3 trips abroad per year for the last 4 years}
\cvitem{Nature}{Love hiking and hanging around in wild places}

\clearpage

\end{document}
